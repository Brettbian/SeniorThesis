\section{Conclusion}
In this paper, we introduce a mathematical formulation of the GAN algorithm in terms of the mean field optimal control theory. Moreover, experiments on imitation of normal distribution and uniform distribution are done to provide empirical understanding on the training strategy. Lastly, the evolution of points under the control of both the discriminator and the generator is visualized to understand the resNet from the perspective of dynamical systems.

\section{Discussion}
As the GAN algorithm is theoretically formulated in the paper in terms of mean field optimal control, we can further apply important results from the optimal control theory like HJB equation and Pontrygin maximum principle to further analyze the problem theoretically. Moreover, as the minimax game structure is embedded in the GAN algorithm, it may be possible to find a relationship between optimal control and the Nash Equilibrium. In the experiments, the results on simple distributions are encouraging and we only experiment on the same family of distribution for both fake and true data distribution. It will be interesting to visualize how generator controls a uniform distribution to imitate a normal distribution. More experiments and visualizations on more complicated distributions may be insightful in the problem. 